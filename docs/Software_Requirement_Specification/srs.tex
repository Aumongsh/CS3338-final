\documentclass[12pt]{article}

% packages
\usepackage[a4paper, margin=1in]{geometry}
\usepackage{fancyhdr}
\usepackage{amsmath}
\usepackage{graphicx}
\usepackage{hyperref}
\usepackage{tabularx}
\usepackage{lipsum}
\usepackage{geometry}
\usepackage{caption}
\usepackage{graphicx}
\usepackage{enumitem}

\pagestyle{fancy}
\fancyhead[L]{Software Requirement Specification}
\fancyfoot[C]{\thepage}

\title{Software Requirement Specification}
\author{Group 6}
\date{Version 4.0}

\begin{document}

% title page
\begin{titlepage}
    \centering
    \vspace*{3cm}
    {\Huge \bfseries Software Requirement Specification \par}
    {\huge Version 4.0 \par}
    \vspace{1cm}
    {\large Group 6 \par}
    {Momoka Aung \par}
    {Alexander Jacobo \par}
    {Tae Ha Kim \par}
    {Kadence Tang \par}
\end{titlepage}

% table of contents
\tableofcontents
\newpage

% version history
\section{Version History}
\begin{tabularx}{\textwidth}{|c|c|c|X|}
    \hline
    \textbf {Editor} & \textbf{Version} & \textbf{Date} & \textbf{Description} \\
    \hline
    Group 6 & 1.0.0 & 4 Dec 2025 & Checkpoint 1: First draft of document. Basic information added for Lab Activity. \\
    \hline
    Kadence Tang & 2.0.0 & 5 Dec 2025 & Checkpoint 2: Added helpful information and more details. \\
    \hline
    Kadence Tang & 3.0.0 & 7 Dec 2025 & Checkpoint 3:  Fixed wording, rewrote Legal and Ethical Considerations section. \\
    \hline
    Kadence Tang & 4.0.0 & 9 Dec 2025 & Final Draft. \\
    \hline 
\end{tabularx}

% sections 
\section{Introduction}
    \subsection{Purpose}
    This SRS outlines the requirements for the Influencer Analysis Tool (IAT). It serves as a guide for developers, managers, and users to understand the scope, functionality, and technical requirements of the system.
    
    \subsection{Intended Audience}
    The intended audience includes the following.
    \begin{itemize}
        \item YouTube Content Creators
        \item Content Creator Managers
        \item Software Developers
        \item Testers
    \end{itemize}
    
    \subsection{Overview}
    The IAT identifies shared viewers between two YouTube channels by scraping comment sections and comparing usernames. It calculates overlap frequency and percentage, presenting results in a clear format to help creators understand their shared audience base.

    \subsection{Product Scope}
        \subsubsection{Product}
        The product is the "Influencer Analysis Tool" web application.

        \subsubsection{Description}
        The IAT enables YouTube creators to input their own channel identifier and a target channel identifier. It scrapes public comment sections, compares usernames, and calculates overlap statistics. Results are displayed in a structured format accessible on both desktop and mobile.

        \subsubsection{Product Objectives}
        The software aims to provide actionable insights into audience overlap, support collaboration decisions, and help creators understand their reach. Objectives include:
        \begin{itemize}
            \item Simplify audience analysis between channels.
            \item Provide secure and accurate overlap statistics.
            \item Ensure compliance with YouTube’s API policies.
        \end{itemize}

        
\section{External Interface Requirements}
    \subsection{User Interface}
    The interface is a simple web form to input channel identifiers. Results are displayed as text, showing the number of overlapping viewers and the percentage of shared viewers. A dashboard view may include tutorials, usage instructions, and feedback options.

    \subsection{Hardware Interfaces}
    Users will need a mobile device or personal computer. On desktop, a keyboard and mouse or trackpad are required. On mobile, touchscreen input will be supported.
    
    \subsection{Software Interfaces}
    The tool communicates with YouTube APIs to scrape public comment data. It will be built using:
   \begin{itemize}
        \item \textbf{Frontend: React 18.2.0} – Provides the user interface framework for building interactive web pages and forms.
        \item \textbf{Backend: Node.js 20.17.0 LTS with Fastify} – Handles server-side logic, routing, and efficient request/response management.
        \item \textbf{YouTube API via Axios} – Axios is used as the HTTP client to communicate with the YouTube Data API, enabling scraping of public comment data.
        \item \textbf{Database: PostgreSQL 18} – Stores channel identifiers, scraped usernames, recurring viewer records, and overlap statistics.
        \item \textbf{JavaScript} – Core programming languages used across both frontend and backend for consistent development.
    \end{itemize}

    \subsection{Communications Interfaces}
    Users may contact the development team via email for support or bug reports. The application will also provide optional notifications or reports via email.

\section{Legal and Ethical Considerations}
    The IAT operates exclusively on public information, specifically usernames that appear in the comment sections of YouTube videos. No other account identifiers or personal data are collected, stored, or used in the analysis.
    \textbf{Privacy:}
    \begin{itemize}
        \item Only public usernames are collected; no sensitive personal data is accessed.
        \item User-provided channel identifiers are stored securely.
        \item Results are anonymized and used solely for overlap analysis.
    \end{itemize}
    
    \textbf{Security:}
    \begin{itemize}
        \item Secure authentication mechanisms will be implemented for user accounts.
        \item Data transfer between client and server will use HTTPS encryption.
        \item Databases will be protected against unauthorized access.
    \end{itemize}
    
    \textbf{Compliance:}
    \begin{itemize}
        \item The IAT adheres to YouTube API Terms of Service.
        \item The tool respects GDPR and COPPA guidelines where applicable.
        \item Clear documentation will inform users about data usage and limitations.
    \end{itemize}
    
    The IAT addresses key legal and ethical concerns by operating only on public YouTube data. The tool complies with YouTube’s API TOS, ensuring privacy, security, and adherence to platform policies.

% glossary
\section{Glossary}
\begin{tabularx}{\textwidth}{|l|X|}
    \hline
    \textbf{Term} & \textbf{Definition} \\
    \hline
    IAT & Influencer Analysis Tool \\
    \hline
    API & Application Programming Interface \\
    \hline
    UI & User Interface \\
    \hline
    TOS & Terms of Service \\
    \hline
    GDPR & General Data Protection Regulation \\
    \hline
    COPPA & Children's Online Privacy Protection Act \\
    \hline
\end{tabularx}

% references
\begin{thebibliography}{9}
    \bibitem{ytAPI} “YouTube API Services Terms of Service  |  Google for Developers.” Google, Google, developers.google.com/youtube/terms/api-services-terms-of-service. Accessed 4 Dec. 2025. 
    % this is a placeholder reference
\end{thebibliography}

\end{document}

